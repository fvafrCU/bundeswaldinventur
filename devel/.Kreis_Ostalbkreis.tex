% This file was generated on Wed May 16 11:01:08 2018 by devel.R.

\section{Eigentumsarten}
Diese Auswertung bezieht sich auf die gesamte Waldfl\"a{}che,
               einschlie\ss{}lich Nichtholzboden. Alle folgenden Auswertungen
               betrachten nur den begehbaren, bestockten Holzboden.

\begin{center}\includegraphics[width=0.8\textwidth]{./devel.R_graphics/Kreis_Ostalbkreis_Eigentumsarten.eps}\end{center}
Hier bezeichnet StW den Staatswald, KW den K\"o{}rperschaftswald, GPW den Gro\ss{}privatwald mit mehr als 200 Hektar und MPW den Mittleren Privatwald mit mehr als 5 und nicht mehr als 200 Hektar und KPW den  Kleinprivatwald mit nicht mehr als 5 Hektar Waldfl\"a{}che.
\newpage
\section{Gesamtwald}
\subsection{Baumartenfl\"a{}chen}
Alle Baumartenfl\"a{}chen enthalten anteilig Bl\"o{}\ss{}en
               und L\"u{}cken.
\subsubsection{Entwicklung von 1987 bis 2012}

\begin{center}\includegraphics{./devel.R_graphics/Kreis_Ostalbkreis_Gesamtwald_Flaeche.eps}\end{center}
% latex table generated in R 3.3.3 by xtable 1.8-2 package
% Wed May 16 11:01:22 2018
\begin{table}[ht]
\centering
\begingroup\scriptsize
\begin{tabular}{l|rr|rr|rr}
  BWI &  \multicolumn{2}{c}{1987} & \multicolumn{2}{c}{2002} & \multicolumn{2}{c}{2012} \\ \toprule
Artengruppe & Fläche & Fehler & Fläche & Fehler & Fläche & Fehler \\ 
  \midrule
                                Fichte & 34.763 & 2.919 & 28.501 & 2.420 & 26.254 & 2.293 \\ 
  Weißtanne/Douglasie/Kiefer/Lärchen/sNB &  7.484 &   969 &  8.682 & 1.085 &  8.639 & 1.055 \\ 
                                   Buche & 11.023 & 1.544 & 12.492 & 1.590 & 13.026 & 1.560 \\ 
                                  Eichen &  2.177 &   364 &  2.360 &   396 &  2.875 &   454 \\ 
            Esche/Bergahorn/HBu/sBlb/Aln &  4.168 &   614 &  7.979 &   959 &  9.841 & 1.038 \\ 
   \midrule
                               Alle BA & 59.614 & 4.407 & 60.014 & 4.428 & 60.635 & 4.411 \\ 
   \bottomrule
\end{tabular}
\endgroup
\caption{Fl\"achen und Fehler in Hektar.} 
\end{table}
% latex table generated in R 3.3.3 by xtable 1.8-2 package
% Wed May 16 11:01:23 2018
\begin{table}[ht]
\centering
\begingroup\scriptsize
\begin{tabular}{l|rr|rr|rr}
  BWI &  \multicolumn{2}{c}{1987} & \multicolumn{2}{c}{2002} & \multicolumn{2}{c}{2012} \\ \toprule
Artengruppe & Anteil & Fehler & Anteil & Fehler & Anteil & Fehler \\ 
  \midrule
                                Fichte & 58,3 & 2,40 & 47,5 & 2,20 & 43,3 & 2,14 \\ 
  Weißtanne/Douglasie/Kiefer/Lärchen/sNB & 12,6 & 1,35 & 14,5 & 1,46 & 14,2 & 1,39 \\ 
                                   Buche & 18,5 & 2,12 & 20,8 & 2,08 & 21,5 & 1,97 \\ 
                                  Eichen &  3,7 & 0,57 &  3,9 & 0,61 &  4,7 & 0,69 \\ 
            Esche/Bergahorn/HBu/sBlb/Aln &  7,0 & 0,92 & 13,3 & 1,28 & 16,2 & 1,28 \\ 
   \bottomrule
\end{tabular}
\endgroup
\caption{Flächenanteil und Fehler in Prozent.} 
\end{table}
\newpage
\subsubsection{Entwicklung der Altersstruktur von 1987 bis 2012}

\subsubsection{Fichte}

\begin{center}\includegraphics{./devel.R_graphics/Kreis_Ostalbkreis_Gesamtwald_Flaeche_Fichte.eps}\end{center}
% latex table generated in R 3.3.3 by xtable 1.8-2 package
% Wed May 16 11:01:25 2018
\begin{table}[ht]
\centering
\begingroup\scriptsize
\begin{tabular}{l|rr|rr|rr}
  BWI &  \multicolumn{2}{c}{1987} & \multicolumn{2}{c}{2002} & \multicolumn{2}{c}{2012} \\ \toprule
Alter & Fläche & Fehler & Fläche & Fehler & Fläche & Fehler \\ 
   1-60 & 17.827 & 1.820 & 18.074 & 1.774 & 17.144 & 1.745 \\ 
  61-120 & 16.428 & 1.794 &  9.479 & 1.164 &  7.891 & 1.010 \\ 
    $>$120 &    508 &   221 &    948 &   263 &  1.219 &   325 \\ 
   \midrule
  Alle & 34.763 & 2.919 & 28.501 & 2.420 & 26.254 & 2.293 \\ 
   \bottomrule
\end{tabular}
\endgroup
\caption{Fl\"achen und Fehler in Hektar.} 
\end{table}

\subsubsection{Weißtanne, Douglasie, Kiefer, Lärchen, Lärchen/Sonstiges Nadelholz}

\begin{center}\includegraphics{./devel.R_graphics/Kreis_Ostalbkreis_Gesamtwald_Flaeche_Weisstanne_Douglasie_Kiefer_Laerchen_sNB.eps}\end{center}
% latex table generated in R 3.3.3 by xtable 1.8-2 package
% Wed May 16 11:01:29 2018
\begin{table}[ht]
\centering
\begingroup\scriptsize
\begin{tabular}{l|rr|rr|rr}
  BWI &  \multicolumn{2}{c}{1987} & \multicolumn{2}{c}{2002} & \multicolumn{2}{c}{2012} \\ \toprule
Alter & Fläche & Fehler & Fläche & Fehler & Fläche & Fehler \\ 
   1-60 & 3.168 & 581 & 3.396 &   585 & 3.832 &   638 \\ 
  61-120 & 3.879 & 620 & 3.993 &   703 & 3.384 &   613 \\ 
    $>$120 &   437 & 177 & 1.292 &   365 & 1.422 &   336 \\ 
   \midrule
  Alle & 7.484 & 969 & 8.682 & 1.085 & 8.639 & 1.055 \\ 
   \bottomrule
\end{tabular}
\endgroup
\caption{Fl\"achen und Fehler in Hektar.} 
\end{table}

\subsubsection{Buche}

\begin{center}\includegraphics{./devel.R_graphics/Kreis_Ostalbkreis_Gesamtwald_Flaeche_Buche.eps}\end{center}
% latex table generated in R 3.3.3 by xtable 1.8-2 package
% Wed May 16 11:01:30 2018
\begin{table}[ht]
\centering
\begingroup\scriptsize
\begin{tabular}{l|rr|rr|rr}
  BWI &  \multicolumn{2}{c}{1987} & \multicolumn{2}{c}{2002} & \multicolumn{2}{c}{2012} \\ \toprule
Alter & Fläche & Fehler & Fläche & Fehler & Fläche & Fehler \\ 
   1-60 &  4.550 &   735 &  5.002 &   818 &  5.416 &   788 \\ 
  61-120 &  5.172 & 1.039 &  5.171 &   896 &  5.253 &   861 \\ 
    $>$120 &  1.302 &   391 &  2.319 &   558 &  2.356 &   539 \\ 
   \midrule
  Alle & 11.023 & 1.544 & 12.492 & 1.590 & 13.026 & 1.560 \\ 
   \bottomrule
\end{tabular}
\endgroup
\caption{Fl\"achen und Fehler in Hektar.} 
\end{table}

\subsubsection{Eichen}

\begin{center}\includegraphics{./devel.R_graphics/Kreis_Ostalbkreis_Gesamtwald_Flaeche_Eichen.eps}\end{center}
% latex table generated in R 3.3.3 by xtable 1.8-2 package
% Wed May 16 11:01:31 2018
\begin{table}[ht]
\centering
\begingroup\scriptsize
\begin{tabular}{l|rr|rr|rr}
  BWI &  \multicolumn{2}{c}{1987} & \multicolumn{2}{c}{2002} & \multicolumn{2}{c}{2012} \\ \toprule
Alter & Fläche & Fehler & Fläche & Fehler & Fläche & Fehler \\ 
   1-60 &   747 & 165 &   915 & 233 & 1.266 & 308 \\ 
  61-120 &   862 & 229 &   808 & 210 &   911 & 214 \\ 
    $>$120 &   568 & 207 &   637 & 196 &   698 & 219 \\ 
   \midrule
  Alle & 2.177 & 364 & 2.360 & 396 & 2.875 & 454 \\ 
   \bottomrule
\end{tabular}
\endgroup
\caption{Fl\"achen und Fehler in Hektar.} 
\end{table}

\subsubsection{Esche, Bergahorn, Hainbuche, Sonstiges Buntlaubholz, Anderes Laubholz niederer Lebensdauer}

\begin{center}\includegraphics{./devel.R_graphics/Kreis_Ostalbkreis_Gesamtwald_Flaeche_Esche_Bergahorn_HBu_sBlb_Aln.eps}\end{center}
% latex table generated in R 3.3.3 by xtable 1.8-2 package
% Wed May 16 11:01:35 2018
\begin{table}[ht]
\centering
\begingroup\scriptsize
\begin{tabular}{l|rr|rr|rr}
  BWI &  \multicolumn{2}{c}{1987} & \multicolumn{2}{c}{2002} & \multicolumn{2}{c}{2012} \\ \toprule
Alter & Fläche & Fehler & Fläche & Fehler & Fläche & Fehler \\ 
   1-60 & 3.116 & 522 & 6.384 & 851 & 8.068 &   940 \\ 
  61-120 & 1.016 & 271 & 1.524 & 321 & 1.610 &   313 \\ 
    $>$120 &    35 &  25 &    71 &  55 &   163 &    77 \\ 
   \midrule
  Alle & 4.168 & 614 & 7.979 & 959 & 9.841 & 1.038 \\ 
   \bottomrule
\end{tabular}
\endgroup
\caption{Fl\"achen und Fehler in Hektar.} 
\end{table}

\subsubsection{Alle Baumarten}

\begin{center}\includegraphics{./devel.R_graphics/Kreis_Ostalbkreis_Gesamtwald_Flaeche_Alle_BA.eps}\end{center}
% latex table generated in R 3.3.3 by xtable 1.8-2 package
% Wed May 16 11:01:36 2018
\begin{table}[ht]
\centering
\begingroup\scriptsize
\begin{tabular}{l|rr|rr|rr}
  BWI &  \multicolumn{2}{c}{1987} & \multicolumn{2}{c}{2002} & \multicolumn{2}{c}{2012} \\ \toprule
Alter & Fläche & Fehler & Fläche & Fehler & Fläche & Fehler \\ 
   1-60 & 29.408 & 2.657 & 33.771 & 2.919 & 35.727 & 3.005 \\ 
  61-120 & 27.357 & 2.586 & 20.975 & 2.122 & 19.049 & 1.964 \\ 
    $>$120 &  2.850 &   611 &  5.268 &   862 &  5.859 &   894 \\ 
   \midrule
  Alle & 59.614 & 4.407 & 60.014 & 4.428 & 60.635 & 4.411 \\ 
   \bottomrule
\end{tabular}
\endgroup
\caption{Fl\"achen und Fehler in Hektar.} 
\end{table}
\newpage
\subsection{Vorr\"ate}
Alle Vorratsangaben umfassen Haupt- und
               Nebenbestandsvorr"a{}te.
\subsubsection{Entwicklung von 1987 bis 2012}

\begin{center}\includegraphics{./devel.R_graphics/Kreis_Ostalbkreis_Gesamtwald_Vorrat.eps}\end{center}
% latex table generated in R 3.3.3 by xtable 1.8-2 package
% Wed May 16 11:01:36 2018
\begin{table}[ht]
\centering
\begingroup\scriptsize
\begin{tabular}{l|rr|rr|rr}
  BWI &  \multicolumn{2}{c}{1987} & \multicolumn{2}{c}{2002} & \multicolumn{2}{c}{2012} \\ \toprule
Artengruppe & Vorrat & Fehler & Vorrat & Fehler & Vorrat & Fehler \\ 
  \midrule
                                    Fichte & 14.469.568 & 1.370.214 & 10.072.072 &   947.826 &  8.989.495 &   873.994 \\ 
  Weißtanne
Douglasie
Kiefer
Lärchen
sNB &  2.630.375 &   355.902 &  3.172.341 &   426.576 &  3.293.193 &   430.848 \\ 
                                       Buche &  3.001.147 &   481.652 &  3.647.308 &   518.080 &  4.216.855 &   586.094 \\ 
                                      Eichen &    574.320 &   100.148 &    665.264 &   114.189 &    891.880 &   141.573 \\ 
            Esche
Bergahorn
HBu
sBlb
Aln &    736.771 &   139.792 &  1.270.219 &   220.144 &  1.766.700 &   253.182 \\ 
   \midrule
                                   Alle BA & 21.412.182 & 1.756.053 & 18.827.204 & 1.536.137 & 19.158.124 & 1.562.268 \\ 
   \bottomrule
\end{tabular}
\endgroup
\caption{Vorr\"ate und Fehler in Kubikmetern Derbholz mit Rinde.} 
\end{table}
% latex table generated in R 3.3.3 by xtable 1.8-2 package
% Wed May 16 11:01:37 2018
\begin{table}[ht]
\centering
\begingroup\scriptsize
\begin{tabular}{l|rr|rr|rr}
  BWI &  \multicolumn{2}{c}{1987} & \multicolumn{2}{c}{2002} & \multicolumn{2}{c}{2012} \\ \toprule
Artengruppe & Vorrat & Fehler & Vorrat & Fehler & Vorrat & Fehler \\ 
  \midrule
                                Fichte & 416 & 16 & 353 & 16 & 342 & 18 \\ 
  Weißtanne/Douglasie/Kiefer/Lärchen/sNB & 351 & 19 & 365 & 26 & 381 & 27 \\ 
                                   Buche & 272 & 17 & 292 & 18 & 324 & 21 \\ 
                                  Eichen & 264 & 23 & 282 & 26 & 310 & 30 \\ 
            Esche/Bergahorn/HBu/sBlb/Aln & 177 & 23 & 159 & 22 & 180 & 19 \\ 
   \midrule
                               Alle BA & 359 & 13 & 314 & 12 & 316 & 13 \\ 
   \bottomrule
\end{tabular}
\endgroup
\caption{Vorr\"ate und Fehler in Kubikmetern Derbholz mit Rinde je Hektar.} 
\end{table}
\newpage
\subsubsection{Entwicklung der Dimensionsstruktur von 1987 bis 2012}

\subsubsection{Fichte}

\begin{center}\includegraphics{./devel.R_graphics/Kreis_Ostalbkreis_Gesamtwald_Vorrat_Fichte.eps}\end{center}
% latex table generated in R 3.3.3 by xtable 1.8-2 package
% Wed May 16 11:01:38 2018
\begin{table}[ht]
\centering
\begingroup\scriptsize
\begin{tabular}{l|rr|rr|rr}
  BWI &  \multicolumn{2}{c}{1987} & \multicolumn{2}{c}{2002} & \multicolumn{2}{c}{2012} \\ \toprule
Durchmesser & Vorrat & Fehler & Vorrat & Fehler & Vorrat & Fehler \\ 
  \midrule
 7-24.9 &  3.359.005 &   384.140 &  2.254.224 & 265.304 & 2.107.777 & 252.680 \\ 
  25-49.9 &  9.745.966 & 1.026.876 &  6.332.239 & 652.115 & 5.402.273 & 589.931 \\ 
     $>$=50 &  1.364.598 &   258.159 &  1.485.608 & 246.080 & 1.479.445 & 227.007 \\ 
   \midrule
   Alle & 14.469.568 & 1.370.214 & 10.072.072 & 947.826 & 8.989.495 & 873.994 \\ 
   \bottomrule
\end{tabular}
\endgroup
\caption{Vorr\"ate und Fehler in Kubikmetern Derbholz mit Rinde.} 
\end{table}

\subsubsection{Weißtanne, Douglasie, Kiefer, Lärchen, Lärchen/Sonstiges Nadelholz}

\begin{center}\includegraphics{./devel.R_graphics/Kreis_Ostalbkreis_Gesamtwald_Vorrat_Weisstanne_Douglasie_Kiefer_Laerchen_sNB.eps}\end{center}
% latex table generated in R 3.3.3 by xtable 1.8-2 package
% Wed May 16 11:01:39 2018
\begin{table}[ht]
\centering
\begingroup\scriptsize
\begin{tabular}{l|rr|rr|rr}
  BWI &  \multicolumn{2}{c}{1987} & \multicolumn{2}{c}{2002} & \multicolumn{2}{c}{2012} \\ \toprule
Durchmesser & Vorrat & Fehler & Vorrat & Fehler & Vorrat & Fehler \\ 
  \midrule
 7-24.9 &   455.884 &  92.078 &   383.892 &  82.400 &   393.664 &  84.097 \\ 
  25-49.9 & 1.858.405 & 268.991 & 2.161.892 & 304.784 & 1.948.709 & 279.985 \\ 
     $>$=50 &   316.086 &  77.661 &   626.557 & 145.900 &   950.819 & 168.652 \\ 
   \midrule
   Alle & 2.630.375 & 355.902 & 3.172.341 & 426.576 & 3.293.193 & 430.848 \\ 
   \bottomrule
\end{tabular}
\endgroup
\caption{Vorr\"ate und Fehler in Kubikmetern Derbholz mit Rinde.} 
\end{table}

\subsubsection{Buche}

\begin{center}\includegraphics{./devel.R_graphics/Kreis_Ostalbkreis_Gesamtwald_Vorrat_Buche.eps}\end{center}
% latex table generated in R 3.3.3 by xtable 1.8-2 package
% Wed May 16 11:01:43 2018
\begin{table}[ht]
\centering
\begingroup\scriptsize
\begin{tabular}{l|rr|rr|rr}
  BWI &  \multicolumn{2}{c}{1987} & \multicolumn{2}{c}{2002} & \multicolumn{2}{c}{2012} \\ \toprule
Durchmesser & Vorrat & Fehler & Vorrat & Fehler & Vorrat & Fehler \\ 
  \midrule
 7-24.9 &   725.811 & 115.923 &   680.600 & 102.938 &   643.168 &  93.596 \\ 
  25-49.9 & 1.679.302 & 312.973 & 1.920.696 & 310.663 & 2.168.687 & 328.713 \\ 
     $>$=50 &   596.034 & 140.549 & 1.046.012 & 223.994 & 1.405.001 & 288.220 \\ 
   \midrule
   Alle & 3.001.147 & 481.652 & 3.647.308 & 518.080 & 4.216.855 & 586.094 \\ 
   \bottomrule
\end{tabular}
\endgroup
\caption{Vorr\"ate und Fehler in Kubikmetern Derbholz mit Rinde.} 
\end{table}

\subsubsection{Eichen}

\begin{center}\includegraphics{./devel.R_graphics/Kreis_Ostalbkreis_Gesamtwald_Vorrat_Eichen.eps}\end{center}
% latex table generated in R 3.3.3 by xtable 1.8-2 package
% Wed May 16 11:01:44 2018
\begin{table}[ht]
\centering
\begingroup\scriptsize
\begin{tabular}{l|rr|rr|rr}
  BWI &  \multicolumn{2}{c}{1987} & \multicolumn{2}{c}{2002} & \multicolumn{2}{c}{2012} \\ \toprule
Durchmesser & Vorrat & Fehler & Vorrat & Fehler & Vorrat & Fehler \\ 
  \midrule
 7-24.9 & 120.278 &  31.534 &  86.167 &  25.034 &  83.814 &  33.681 \\ 
  25-49.9 & 305.535 &  72.609 & 339.352 &  69.780 & 418.203 &  79.123 \\ 
     $>$=50 & 148.507 &  41.919 & 239.744 &  56.302 & 389.864 &  80.254 \\ 
   \midrule
   Alle & 574.320 & 100.148 & 665.264 & 114.189 & 891.880 & 141.573 \\ 
   \bottomrule
\end{tabular}
\endgroup
\caption{Vorr\"ate und Fehler in Kubikmetern Derbholz mit Rinde.} 
\end{table}

\subsubsection{Esche, Bergahorn, Hainbuche, Sonstiges Buntlaubholz, Anderes Laubholz niederer Lebensdauer}

\begin{center}\includegraphics{./devel.R_graphics/Kreis_Ostalbkreis_Gesamtwald_Vorrat_Esche_Bergahorn_HBu_sBlb_Aln.eps}\end{center}
% latex table generated in R 3.3.3 by xtable 1.8-2 package
% Wed May 16 11:01:45 2018
\begin{table}[ht]
\centering
\begingroup\scriptsize
\begin{tabular}{l|rr|rr|rr}
  BWI &  \multicolumn{2}{c}{1987} & \multicolumn{2}{c}{2002} & \multicolumn{2}{c}{2012} \\ \toprule
Durchmesser & Vorrat & Fehler & Vorrat & Fehler & Vorrat & Fehler \\ 
  \midrule
 7-24.9 & 248.466 &  52.901 &   394.413 &  76.635 &   581.241 &  91.903 \\ 
  25-49.9 & 418.494 &  91.001 &   716.730 & 145.945 &   906.670 & 159.881 \\ 
     $>$=50 &  69.811 &  28.387 &   159.076 &  52.500 &   278.789 &  70.241 \\ 
   \midrule
   Alle & 736.771 & 139.792 & 1.270.219 & 220.144 & 1.766.700 & 253.182 \\ 
   \bottomrule
\end{tabular}
\endgroup
\caption{Vorr\"ate und Fehler in Kubikmetern Derbholz mit Rinde.} 
\end{table}

\subsubsection{Alle Baumarten}

\begin{center}\includegraphics{./devel.R_graphics/Kreis_Ostalbkreis_Gesamtwald_Vorrat_Alle_BA.eps}\end{center}
% latex table generated in R 3.3.3 by xtable 1.8-2 package
% Wed May 16 11:01:49 2018
\begin{table}[ht]
\centering
\begingroup\scriptsize
\begin{tabular}{l|rr|rr|rr}
  BWI &  \multicolumn{2}{c}{1987} & \multicolumn{2}{c}{2002} & \multicolumn{2}{c}{2012} \\ \toprule
Durchmesser & Vorrat & Fehler & Vorrat & Fehler & Vorrat & Fehler \\ 
  \midrule
 7-24.9 &  4.909.445 &   472.902 &  3.799.297 &   353.559 &  3.809.664 &   352.892 \\ 
  25-49.9 & 14.007.701 & 1.260.041 & 11.470.910 & 1.007.485 & 10.844.542 &   962.154 \\ 
     $>$=50 &  2.495.036 &   342.826 &  3.556.997 &   442.031 &  4.503.918 &   524.993 \\ 
   \midrule
   Alle & 21.412.182 & 1.756.053 & 18.827.204 & 1.536.137 & 19.158.124 & 1.562.268 \\ 
   \bottomrule
\end{tabular}
\endgroup
\caption{Vorr\"ate und Fehler in Kubikmetern Derbholz mit Rinde.} 
\end{table}
\newpage
\subsection{Naturn"a{}he}
Bei der Bestimmung der Naturn"a{}he wurde der Vergleichbarkeit
               wegen f"u{}r BWI 2 und BWI 3 die Zuordnung der
               nat"u{}rlichen Waldgesellschaft der BWI 3 zugrundegelegt.

\begin{center}\includegraphics{./devel.R_graphics/Kreis_Ostalbkreis_Gesamtwald_Naturnaehe.eps}\end{center}
% latex table generated in R 3.3.3 by xtable 1.8-2 package
% Wed May 16 11:01:51 2018
\begin{table}[ht]
\centering
\begingroup\scriptsize
\begin{tabular}{l|rr|rr}
  &  \multicolumn{2}{c}{2002} & \multicolumn{2}{c}{2012} \\ \toprule
Naturnaehestufe & Fl"achenanteil & Fehler & Fl"achenanteil & Fehler \\ 
  \midrule
   sehr naturnah & 17,3 & 2,1 & 17,2 & 2,1 \\ 
          naturnah & 30,7 & 2,1 & 30,7 & 2,1 \\ 
  bedingt naturnah & 35,0 & 2,3 & 35,3 & 2,3 \\ 
      kulturbetont &  4,3 & 1,0 &  4,5 & 1,0 \\ 
    kulturbestimmt & 11,8 & 1,9 & 12,0 & 1,9 \\ 
   \midrule
    keine Angabe &  0,8 & 0,4 &  0,3 & 0,2 \\ 
   \bottomrule
\end{tabular}
\endgroup
\caption{Naturn\"ahestufen in der Fl\"ache und Fehler in Prozent.} 
\end{table}
\subsection{Forstlich bedeutsame Arten}

\begin{center}\includegraphics{./devel.R_graphics/Kreis_Ostalbkreis_Gesamtwald_Forstlich_bedeutsame_Arten.eps}\end{center}
% latex table generated in R 3.3.3 by xtable 1.8-2 package
% Wed May 16 11:01:53 2018
\begin{table}[ht]
\centering
\begingroup\scriptsize
\begin{tabular}{l|rr|rr}
  BWI &  \multicolumn{2}{c}{2002} & \multicolumn{2}{c}{2012} \\ \toprule
Art & Flächenanteil & Fehler & Flächenanteil & Fehler \\ 
  \midrule
  Adlerfarn & 0,0 & 0,00 & 0,0 & 0,03 \\ 
   Brennessel & 3,6 & 0,49 & 2,9 & 0,35 \\ 
    Brombeere & 7,3 & 0,87 & 8,2 & 0,86 \\ 
   Heidekraut & 0,0 & 0,00 & 0,1 & 0,02 \\ 
  Heidelbeere & 3,3 & 0,56 & 4,3 & 0,73 \\ 
    Honiggras & 0,0 & 0,02 & 0,3 & 0,10 \\ 
     Reitgras & 0,6 & 0,21 & 2,1 & 0,37 \\ 
     Riedgras & 3,3 & 0,66 & 5,6 & 1,00 \\ 
   \bottomrule
\end{tabular}
\endgroup
\caption{Forstlich bedeutsame Arten: Anteile und Fehler in Prozent.} 
\end{table}
\clearpage
\subsection{Totholz}

\begin{center}\includegraphics[width=1\textwidth]{./devel.R_graphics/Kreis_Ostalbkreis_Gesamtwald_Totholz.eps}\end{center}
% latex table generated in R 3.3.3 by xtable 1.8-2 package
% Wed May 16 11:01:58 2018
\begin{table}[ht]
\centering
\begingroup\scriptsize
\begin{tabular}{l|rr|rr|rr}
  BWI &  \multicolumn{2}{c}{2002} & \multicolumn{2}{c}{2012*} & \multicolumn{2}{c}{2012} \\ \toprule
Totholzart & Vorrat & Fehler & Vorrat & Fehler & Vorrat & Fehler \\ 
  \midrule
Liegend & 8.5 & 1.4 & 8.3 & 1.5 & 11.7 & 1.6 \\ 
  Stehend (Baum) & 0.9 & 0.2 & 1.0 & 0.4 & 1.3 & 0.6 \\ 
  Stehend (Bruchstück) & 0.9 & 0.2 & 1.7 & 0.6 & 2.1 & 0.6 \\ 
  Wurzelstock & 5.9 & 0.5 & 6.2 & 0.5 & 9.6 & 0.6 \\ 
  Abfuhrrest & 0.0 & 0.0 & 0.0 &  & 0.0 &  \\ 
   \midrule
Gesamtvorrat & 16.2 & 1.9 & 17.2 & 1.9 & 24.7 & 2.0 \\ 
   \bottomrule
\end{tabular}
\endgroup
\caption{Totholz: Vorr\"ate und Fehler in Kubikmetern Derbholz mit Rinde je Hektar,
                   2012*:  Berechnung des Totholzes 2012 nach Aufnahmekriterien von 2002.} 
\end{table}
\newpage
\subsection{Zuwachs und ausgeschiedener Vorrat}
Ausgeschiedener Vorrat umfasst genutzte und ungenutzt im Wald
               verbliebene Mengen.

\begin{center}\includegraphics[width=1\textwidth]{./devel.R_graphics/Kreis_Ostalbkreis_Gesamtwald_Zuwachs_Abgang.eps}\end{center}
% latex table generated in R 3.3.3 by xtable 1.8-2 package
% Wed May 16 11:02:07 2018
\begin{table}[ht]
\centering
\begingroup\scriptsize
\begin{tabular}{ll|rr|rr|rr|rr|rr}
  &  \multicolumn{2}{c}{} \\ \toprule
Artengruppe & Periode & Zuwachs & Fehler & Abgang & Fehler \\ 
  \midrule
                                Fichte & 1987 - 2002 & 15,6 & 0,44 & 25,0 & 1,76 \\ 
                                  Fichte & 2002 - 2012 & 13,2 & 0,50 & 18,0 & 1,33 \\ 
  Weißtanne/Douglasie/Kiefer/Lärchen/sNB & 1987 - 2002 & 12,5 & 0,79 &  8,9 & 1,08 \\ 
  Weißtanne/Douglasie/Kiefer/Lärchen/sNB & 2002 - 2012 & 11,5 & 0,88 & 11,3 & 1,49 \\ 
                                   Buche & 1987 - 2002 & 11,4 & 0,47 &  7,0 & 0,90 \\ 
                                   Buche & 2002 - 2012 & 11,3 & 0,58 &  7,8 & 1,01 \\ 
                                  Eichen & 1987 - 2002 &  8,7 & 0,76 &  5,6 & 1,05 \\ 
                                  Eichen & 2002 - 2012 &  9,8 & 0,97 &  4,6 & 1,02 \\ 
            Esche/Bergahorn/HBu/sBlb/Aln & 1987 - 2002 &  6,8 & 0,69 &  2,2 & 0,48 \\ 
            Esche/Bergahorn/HBu/sBlb/Aln & 2002 - 2012 &  7,2 & 0,49 &  3,0 & 0,61 \\ 
   \midrule
                               Alle BA & 1987 - 2002 & 13,2 & 0,33 & 16,3 & 1,06 \\ 
                                 Alle BA & 2002 - 2012 & 11,5 & 0,33 & 12,2 & 0,75 \\ 
   \bottomrule
\end{tabular}
\endgroup
\caption{Zuwachs, ausgeschiedener Vorrat (Abgang) und Fehler in Kubikmeternn Derbholz mit Rinde pro Hektar und Jahr.} 
\end{table}

\begin{center}\includegraphics{./devel.R_graphics/Kreis_Ostalbkreis_Gesamtwald_Nutzungen.eps}\end{center}
% latex table generated in R 3.3.3 by xtable 1.8-2 package
% Wed May 16 11:02:18 2018
\begin{table}[ht]
\centering
\begingroup\scriptsize
\begin{tabular}{l|rr|rr|rr|rr}
  Durchmesser &  \multicolumn{2}{c}{>=50 cm} & \multicolumn{2}{c}{0-24.9 cm} & \multicolumn{2}{c}{25-49.9 cm} & \multicolumn{2}{c}{Gesamt} \\ \toprule
Periode & Nutzung & Fehler & Nutzung & Fehler & Nutzung & Fehler & Nutzung & Fehler \\ 
  \midrule
1987 - 2002 & 173.777 & 29.348 & 206.069 & 24.528 & 723.722 & 79.842 & 1.103.567 & 110.069 \\ 
   \midrule
2002 - 2012 & 133.175 & 20.230 & 100.149 & 12.561 & 350.994 & 36.561 &   584.318 &  55.471 \\ 
   \bottomrule
\end{tabular}
\endgroup
\caption{J\"a{}hrliche Ernte in Kubikmetern Erntevolumen ohne Rinde 
                  nach St\"a{}rkeklassen \"u{}ber alle Baumarten.} 
\end{table}
\subsection{Verj"ungung}
Als Verj"u{}ngung wird die Baumschicht bis zu einer H"o{}he
               von vier Metern betrachtet.
% latex table generated in R 3.3.3 by xtable 1.8-2 package
% Wed May 16 11:02:23 2018
\begin{table}[ht]
\centering
\begingroup\scriptsize
\begin{tabular}{lrrrrrr}
  \hline
BWI & Artengruppe & Unter Schirm & Abgedeckt & Gesamt & Fehler & NV Anteil \\ 
  \hline
2002 &                                 Fichte &  2.827,7 & 2.900 &  5.727 &   726,9 & 72,2 \\ 
  2002 & Weißtanne/Douglasie/Kiefer/Lärchen/sNB &    528,1 &   293 &    821 &   150,1 & 70,6 \\ 
  2002 &                                  Buche &  3.053,7 & 1.439 &  4.493 &   715,1 & 82,1 \\ 
  2002 &                                 Eichen &     21,0 &   251 &    272 &   108,2 & 11,8 \\ 
  2002 &           Esche/Bergahorn/HBu/sBlb/Aln &  3.021,7 & 1.828 &  4.850 &   615,8 & 77,7 \\ 
  2002 &                             Alle Arten &  9.452,3 & 6.712 & 16.164 & 1.566,8 & 75,5 \\ 
  2012 &                                 Fichte &  2.866,6 & 2.076 &  4.943 &   638,7 & 71,6 \\ 
  2012 & Weißtanne/Douglasie/Kiefer/Lärchen/sNB &  1.031,6 &   484 &  1.516 &   251,6 & 64,2 \\ 
  2012 &                                  Buche &  3.834,2 &   829 &  4.664 &   668,5 & 95,0 \\ 
  2012 &                                 Eichen &     91,1 &   102 &    193 &    58,7 & 47,2 \\ 
  2012 &           Esche/Bergahorn/HBu/sBlb/Aln &  3.322,9 & 1.171 &  4.494 &   556,7 & 87,3 \\ 
  2012 &                             Alle Arten & 11.146,4 & 4.663 & 15.809 & 1.487,7 & 82,0 \\ 
   \hline
\end{tabular}
\endgroup
\caption{Verj\"ungung in Hektar, Anteil der Naturverj\"ungung 
                  (NV Anteil) an der gesamten Naturverj\"ungung (Gesamt)
                  in Prozent.} 
\end{table}
\newpage
\section{Privatwald}
\subsection{Baumartenfl\"a{}chen}
Alle Baumartenfl\"a{}chen enthalten anteilig Bl\"o{}\ss{}en
               und L\"u{}cken.
\subsubsection{Entwicklung von 1987 bis 2012}

\begin{center}\includegraphics{./devel.R_graphics/Kreis_Ostalbkreis_Privatwald_Flaeche.eps}\end{center}
% latex table generated in R 3.3.3 by xtable 1.8-2 package
% Wed May 16 11:02:35 2018
\begin{table}[ht]
\centering
\begingroup\scriptsize
\begin{tabular}{l|rr|rr|rr}
  BWI &  \multicolumn{2}{c}{1987} & \multicolumn{2}{c}{2002} & \multicolumn{2}{c}{2012} \\ \toprule
Artengruppe & Fläche & Fehler & Fläche & Fehler & Fläche & Fehler \\ 
  \midrule
                                Fichte & 15.794 & 1.886 & 13.049 & 1.535 & 13.058 & 1.540 \\ 
  Weißtanne/Douglasie/Kiefer/Lärchen/sNB &  3.500 &   717 &  3.745 &   729 &  3.784 &   733 \\ 
                                   Buche &  5.498 & 1.118 &  5.851 & 1.132 &  5.436 &   987 \\ 
                                  Eichen &  1.439 &   321 &  1.211 &   299 &  1.499 &   348 \\ 
            Esche/Bergahorn/HBu/sBlb/Aln &  2.175 &   459 &  4.450 &   739 &  5.041 &   767 \\ 
   \midrule
                               Alle BA & 28.407 & 2.853 & 28.307 & 2.862 & 28.817 & 2.838 \\ 
   \bottomrule
\end{tabular}
\endgroup
\caption{Fl\"achen und Fehler in Hektar.} 
\end{table}
% latex table generated in R 3.3.3 by xtable 1.8-2 package
% Wed May 16 11:02:36 2018
\begin{table}[ht]
\centering
\begingroup\scriptsize
\begin{tabular}{l|rr|rr|rr}
  BWI &  \multicolumn{2}{c}{1987} & \multicolumn{2}{c}{2002} & \multicolumn{2}{c}{2012} \\ \toprule
Artengruppe & Anteil & Fehler & Anteil & Fehler & Anteil & Fehler \\ 
  \midrule
                                Fichte & 55,6 & 3,8 & 46,1 & 3,37 & 45,3 & 3,2 \\ 
  Weißtanne/Douglasie/Kiefer/Lärchen/sNB & 12,3 & 2,2 & 13,2 & 2,24 & 13,1 & 2,2 \\ 
                                   Buche & 19,4 & 3,2 & 20,7 & 3,25 & 18,9 & 2,8 \\ 
                                  Eichen &  5,1 & 1,0 &  4,3 & 0,98 &  5,2 & 1,1 \\ 
            Esche/Bergahorn/HBu/sBlb/Aln &  7,7 & 1,5 & 15,7 & 2,04 & 17,5 & 2,0 \\ 
   \bottomrule
\end{tabular}
\endgroup
\caption{Flächenanteil und Fehler in Prozent.} 
\end{table}
\newpage
\subsubsection{Entwicklung der Altersstruktur von 1987 bis 2012}

\subsubsection{Fichte}

\begin{center}\includegraphics{./devel.R_graphics/Kreis_Ostalbkreis_Privatwald_Flaeche_Fichte.eps}\end{center}
% latex table generated in R 3.3.3 by xtable 1.8-2 package
% Wed May 16 11:02:37 2018
\begin{table}[ht]
\centering
\begingroup\scriptsize
\begin{tabular}{l|rr|rr|rr}
  BWI &  \multicolumn{2}{c}{1987} & \multicolumn{2}{c}{2002} & \multicolumn{2}{c}{2012} \\ \toprule
Alter & Fläche & Fehler & Fläche & Fehler & Fläche & Fehler \\ 
   1-60 &  8.334 & 1.188 &  8.276 & 1.172 &  9.793 & 1.312 \\ 
  61-120 &  7.424 & 1.165 &  4.261 &   724 &  2.925 &   537 \\ 
    $>$120 &     36 &    30 &    512 &   178 &    339 &   131 \\ 
   \midrule
  Alle & 15.794 & 1.886 & 13.049 & 1.535 & 13.058 & 1.540 \\ 
   \bottomrule
\end{tabular}
\endgroup
\caption{Fl\"achen und Fehler in Hektar.} 
\end{table}

\subsubsection{Weißtanne, Douglasie, Kiefer, Lärchen, Lärchen/Sonstiges Nadelholz}

\begin{center}\includegraphics{./devel.R_graphics/Kreis_Ostalbkreis_Privatwald_Flaeche_Weisstanne_Douglasie_Kiefer_Laerchen_sNB.eps}\end{center}
% latex table generated in R 3.3.3 by xtable 1.8-2 package
% Wed May 16 11:02:41 2018
\begin{table}[ht]
\centering
\begingroup\scriptsize
\begin{tabular}{l|rr|rr|rr}
  BWI &  \multicolumn{2}{c}{1987} & \multicolumn{2}{c}{2002} & \multicolumn{2}{c}{2012} \\ \toprule
Alter & Fläche & Fehler & Fläche & Fehler & Fläche & Fehler \\ 
   1-60 & 1.230 & 393 &   987 & 276 & 1.600 & 427 \\ 
  61-120 & 2.197 & 506 & 1.895 & 496 & 1.495 & 426 \\ 
    $>$120 &    72 &  52 &   863 & 318 &   689 & 248 \\ 
   \midrule
  Alle & 3.500 & 717 & 3.745 & 729 & 3.784 & 733 \\ 
   \bottomrule
\end{tabular}
\endgroup
\caption{Fl\"achen und Fehler in Hektar.} 
\end{table}

\subsubsection{Buche}

\begin{center}\includegraphics{./devel.R_graphics/Kreis_Ostalbkreis_Privatwald_Flaeche_Buche.eps}\end{center}
% latex table generated in R 3.3.3 by xtable 1.8-2 package
% Wed May 16 11:02:42 2018
\begin{table}[ht]
\centering
\begingroup\scriptsize
\begin{tabular}{l|rr|rr|rr}
  BWI &  \multicolumn{2}{c}{1987} & \multicolumn{2}{c}{2002} & \multicolumn{2}{c}{2012} \\ \toprule
Alter & Fläche & Fehler & Fläche & Fehler & Fläche & Fehler \\ 
   1-60 & 1.881 &   505 & 1.960 &   585 & 1.963 & 483 \\ 
  61-120 & 2.881 &   769 & 2.471 &   629 & 2.024 & 470 \\ 
    $>$120 &   736 &   274 & 1.419 &   459 & 1.449 & 444 \\ 
   \midrule
  Alle & 5.498 & 1.118 & 5.851 & 1.132 & 5.436 & 987 \\ 
   \bottomrule
\end{tabular}
\endgroup
\caption{Fl\"achen und Fehler in Hektar.} 
\end{table}

\subsubsection{Eichen}

\begin{center}\includegraphics{./devel.R_graphics/Kreis_Ostalbkreis_Privatwald_Flaeche_Eichen.eps}\end{center}
% latex table generated in R 3.3.3 by xtable 1.8-2 package
% Wed May 16 11:02:43 2018
\begin{table}[ht]
\centering
\begingroup\scriptsize
\begin{tabular}{l|rr|rr|rr}
  BWI &  \multicolumn{2}{c}{1987} & \multicolumn{2}{c}{2002} & \multicolumn{2}{c}{2012} \\ \toprule
Alter & Fläche & Fehler & Fläche & Fehler & Fläche & Fehler \\ 
   1-60 &   282 &  93 &   313 & 126 &   533 & 208 \\ 
  61-120 &   668 & 212 &   479 & 178 &   408 & 148 \\ 
    $>$120 &   489 & 199 &   419 & 166 &   558 & 208 \\ 
   \midrule
  Alle & 1.439 & 321 & 1.211 & 299 & 1.499 & 348 \\ 
   \bottomrule
\end{tabular}
\endgroup
\caption{Fl\"achen und Fehler in Hektar.} 
\end{table}

\subsubsection{Esche, Bergahorn, Hainbuche, Sonstiges Buntlaubholz, Anderes Laubholz niederer Lebensdauer}

\begin{center}\includegraphics{./devel.R_graphics/Kreis_Ostalbkreis_Privatwald_Flaeche_Esche_Bergahorn_HBu_sBlb_Aln.eps}\end{center}
% latex table generated in R 3.3.3 by xtable 1.8-2 package
% Wed May 16 11:02:47 2018
\begin{table}[ht]
\centering
\begingroup\scriptsize
\begin{tabular}{l|rr|rr|rr}
  BWI &  \multicolumn{2}{c}{1987} & \multicolumn{2}{c}{2002} & \multicolumn{2}{c}{2012} \\ \toprule
Alter & Fläche & Fehler & Fläche & Fehler & Fläche & Fehler \\ 
   1-60 & 1.517 & 394 & 3.330 & 656 & 4.072 & 708 \\ 
  61-120 &   642 & 182 & 1.058 & 254 &   873 & 195 \\ 
    $>$120 &    16 &  16 &    62 &  55 &    96 &  68 \\ 
   \midrule
  Alle & 2.175 & 459 & 4.450 & 739 & 5.041 & 767 \\ 
   \bottomrule
\end{tabular}
\endgroup
\caption{Fl\"achen und Fehler in Hektar.} 
\end{table}

\subsubsection{Alle Baumarten}

\begin{center}\includegraphics{./devel.R_graphics/Kreis_Ostalbkreis_Privatwald_Flaeche_Alle_BA.eps}\end{center}
% latex table generated in R 3.3.3 by xtable 1.8-2 package
% Wed May 16 11:02:48 2018
\begin{table}[ht]
\centering
\begingroup\scriptsize
\begin{tabular}{l|rr|rr|rr}
  BWI &  \multicolumn{2}{c}{1987} & \multicolumn{2}{c}{2002} & \multicolumn{2}{c}{2012} \\ \toprule
Alter & Fläche & Fehler & Fläche & Fehler & Fläche & Fehler \\ 
   1-60 & 13.244 & 1.663 & 14.867 & 1.827 & 17.961 & 2.046 \\ 
  61-120 & 13.813 & 1.750 & 10.164 & 1.384 &  7.725 & 1.137 \\ 
    $>$120 &  1.350 &   389 &  3.276 &   729 &  3.131 &   690 \\ 
   \midrule
  Alle & 28.407 & 2.853 & 28.307 & 2.862 & 28.817 & 2.838 \\ 
   \bottomrule
\end{tabular}
\endgroup
\caption{Fl\"achen und Fehler in Hektar.} 
\end{table}
\newpage
\subsection{Vorr\"ate}
Alle Vorratsangaben umfassen Haupt- und
               Nebenbestandsvorr"a{}te.
\subsubsection{Entwicklung von 1987 bis 2012}

\begin{center}\includegraphics{./devel.R_graphics/Kreis_Ostalbkreis_Privatwald_Vorrat.eps}\end{center}
% latex table generated in R 3.3.3 by xtable 1.8-2 package
% Wed May 16 11:02:49 2018
\begin{table}[ht]
\centering
\begingroup\scriptsize
\begin{tabular}{l|rr|rr|rr}
  BWI &  \multicolumn{2}{c}{1987} & \multicolumn{2}{c}{2002} & \multicolumn{2}{c}{2012} \\ \toprule
Artengruppe & Vorrat & Fehler & Vorrat & Fehler & Vorrat & Fehler \\ 
  \midrule
                                    Fichte &  6.512.362 &   880.389 & 4.630.005 &   605.418 & 4.379.116 &   579.094 \\ 
  Weißtanne
Douglasie
Kiefer
Lärchen
sNB &  1.321.670 &   282.609 & 1.599.820 &   343.778 & 1.642.703 &   344.576 \\ 
                                       Buche &  1.648.724 &   383.732 & 1.771.069 &   380.449 & 1.884.780 &   420.210 \\ 
                                      Eichen &    364.040 &    84.117 &   398.457 &    95.395 &   490.669 &   111.745 \\ 
            Esche
Bergahorn
HBu
sBlb
Aln &    466.900 &   107.614 &   741.646 &   157.957 &   946.665 &   168.888 \\ 
   \midrule
                                   Alle BA & 10.313.696 & 1.170.469 & 9.140.997 & 1.040.259 & 9.343.933 & 1.072.634 \\ 
   \bottomrule
\end{tabular}
\endgroup
\caption{Vorr\"ate und Fehler in Kubikmetern Derbholz mit Rinde.} 
\end{table}
% latex table generated in R 3.3.3 by xtable 1.8-2 package
% Wed May 16 11:02:49 2018
\begin{table}[ht]
\centering
\begingroup\scriptsize
\begin{tabular}{l|rr|rr|rr}
  BWI &  \multicolumn{2}{c}{1987} & \multicolumn{2}{c}{2002} & \multicolumn{2}{c}{2012} \\ \toprule
Artengruppe & Vorrat & Fehler & Vorrat & Fehler & Vorrat & Fehler \\ 
  \midrule
                                Fichte & 412 & 24 & 355 & 26 & 335 & 29 \\ 
  Weißtanne/Douglasie/Kiefer/Lärchen/sNB & 378 & 33 & 427 & 40 & 434 & 51 \\ 
                                   Buche & 300 & 29 & 303 & 32 & 347 & 42 \\ 
                                  Eichen & 253 & 31 & 329 & 30 & 327 & 50 \\ 
            Esche/Bergahorn/HBu/sBlb/Aln & 215 & 33 & 167 & 30 & 188 & 26 \\ 
   \midrule
                               Alle BA & 363 & 18 & 323 & 19 & 324 & 21 \\ 
   \bottomrule
\end{tabular}
\endgroup
\caption{Vorr\"ate und Fehler in Kubikmetern Derbholz mit Rinde je Hektar.} 
\end{table}
\newpage
\subsubsection{Entwicklung der Dimensionsstruktur von 1987 bis 2012}

\subsubsection{Fichte}

\begin{center}\includegraphics{./devel.R_graphics/Kreis_Ostalbkreis_Privatwald_Vorrat_Fichte.eps}\end{center}
% latex table generated in R 3.3.3 by xtable 1.8-2 package
% Wed May 16 11:02:50 2018
\begin{table}[ht]
\centering
\begingroup\scriptsize
\begin{tabular}{l|rr|rr|rr}
  BWI &  \multicolumn{2}{c}{1987} & \multicolumn{2}{c}{2002} & \multicolumn{2}{c}{2012} \\ \toprule
Durchmesser & Vorrat & Fehler & Vorrat & Fehler & Vorrat & Fehler \\ 
  \midrule
 7-24.9 & 1.580.659 & 249.357 & 1.206.533 & 203.668 & 1.340.654 & 200.986 \\ 
  25-49.9 & 4.278.887 & 657.243 & 2.745.992 & 402.621 & 2.335.710 & 361.702 \\ 
     $>$=50 &   652.816 & 208.924 &   677.480 & 161.357 &   702.752 & 159.726 \\ 
   \midrule
   Alle & 6.512.362 & 880.389 & 4.630.005 & 605.418 & 4.379.116 & 579.094 \\ 
   \bottomrule
\end{tabular}
\endgroup
\caption{Vorr\"ate und Fehler in Kubikmetern Derbholz mit Rinde.} 
\end{table}

\subsubsection{Weißtanne, Douglasie, Kiefer, Lärchen, Lärchen/Sonstiges Nadelholz}

\begin{center}\includegraphics{./devel.R_graphics/Kreis_Ostalbkreis_Privatwald_Vorrat_Weisstanne_Douglasie_Kiefer_Laerchen_sNB.eps}\end{center}
% latex table generated in R 3.3.3 by xtable 1.8-2 package
% Wed May 16 11:02:51 2018
\begin{table}[ht]
\centering
\begingroup\scriptsize
\begin{tabular}{l|rr|rr|rr}
  BWI &  \multicolumn{2}{c}{1987} & \multicolumn{2}{c}{2002} & \multicolumn{2}{c}{2012} \\ \toprule
Durchmesser & Vorrat & Fehler & Vorrat & Fehler & Vorrat & Fehler \\ 
  \midrule
 7-24.9 &   199.885 &  64.570 &   202.338 &  66.292 &   204.177 &  59.107 \\ 
  25-49.9 &   965.480 & 217.058 & 1.104.040 & 239.853 & 1.019.141 & 225.525 \\ 
     $>$=50 &   156.305 &  66.651 &   293.441 & 128.467 &   419.385 & 132.066 \\ 
   \midrule
   Alle & 1.321.670 & 282.609 & 1.599.820 & 343.778 & 1.642.703 & 344.576 \\ 
   \bottomrule
\end{tabular}
\endgroup
\caption{Vorr\"ate und Fehler in Kubikmetern Derbholz mit Rinde.} 
\end{table}

\subsubsection{Buche}

\begin{center}\includegraphics{./devel.R_graphics/Kreis_Ostalbkreis_Privatwald_Vorrat_Buche.eps}\end{center}
% latex table generated in R 3.3.3 by xtable 1.8-2 package
% Wed May 16 11:02:56 2018
\begin{table}[ht]
\centering
\begingroup\scriptsize
\begin{tabular}{l|rr|rr|rr}
  BWI &  \multicolumn{2}{c}{1987} & \multicolumn{2}{c}{2002} & \multicolumn{2}{c}{2012} \\ \toprule
Durchmesser & Vorrat & Fehler & Vorrat & Fehler & Vorrat & Fehler \\ 
  \midrule
 7-24.9 &   335.042 &  82.877 &   245.791 &  59.425 &   237.139 &  54.787 \\ 
  25-49.9 &   990.113 & 259.623 & 1.018.557 & 230.718 & 1.006.760 & 237.093 \\ 
     $>$=50 &   323.570 & 100.487 &   506.720 & 162.805 &   640.881 & 200.868 \\ 
   \midrule
   Alle & 1.648.724 & 383.732 & 1.771.069 & 380.449 & 1.884.780 & 420.210 \\ 
   \bottomrule
\end{tabular}
\endgroup
\caption{Vorr\"ate und Fehler in Kubikmetern Derbholz mit Rinde.} 
\end{table}

\subsubsection{Eichen}

\begin{center}\includegraphics{./devel.R_graphics/Kreis_Ostalbkreis_Privatwald_Vorrat_Eichen.eps}\end{center}
% latex table generated in R 3.3.3 by xtable 1.8-2 package
% Wed May 16 11:02:57 2018
\begin{table}[ht]
\centering
\begingroup\scriptsize
\begin{tabular}{l|rr|rr|rr}
  BWI &  \multicolumn{2}{c}{1987} & \multicolumn{2}{c}{2002} & \multicolumn{2}{c}{2012} \\ \toprule
Durchmesser & Vorrat & Fehler & Vorrat & Fehler & Vorrat & Fehler \\ 
  \midrule
 7-24.9 &  56.999 & 18.881 &  42.721 & 15.701 &  22.282 &   8.260 \\ 
  25-49.9 & 209.060 & 64.307 & 214.448 & 61.597 & 212.057 &  60.731 \\ 
     $>$=50 &  97.982 & 32.510 & 141.288 & 42.716 & 256.330 &  65.641 \\ 
   \midrule
   Alle & 364.040 & 84.117 & 398.457 & 95.395 & 490.669 & 111.745 \\ 
   \bottomrule
\end{tabular}
\endgroup
\caption{Vorr\"ate und Fehler in Kubikmetern Derbholz mit Rinde.} 
\end{table}

\subsubsection{Esche, Bergahorn, Hainbuche, Sonstiges Buntlaubholz, Anderes Laubholz niederer Lebensdauer}

\begin{center}\includegraphics{./devel.R_graphics/Kreis_Ostalbkreis_Privatwald_Vorrat_Esche_Bergahorn_HBu_sBlb_Aln.eps}\end{center}
% latex table generated in R 3.3.3 by xtable 1.8-2 package
% Wed May 16 11:02:58 2018
\begin{table}[ht]
\centering
\begingroup\scriptsize
\begin{tabular}{l|rr|rr|rr}
  BWI &  \multicolumn{2}{c}{1987} & \multicolumn{2}{c}{2002} & \multicolumn{2}{c}{2012} \\ \toprule
Durchmesser & Vorrat & Fehler & Vorrat & Fehler & Vorrat & Fehler \\ 
  \midrule
 7-24.9 & 160.860 &  46.687 & 194.635 &  57.186 & 279.899 &  53.955 \\ 
  25-49.9 & 259.770 &  63.842 & 433.716 & 107.909 & 478.679 &  99.346 \\ 
     $>$=50 &  46.270 &  21.044 & 113.296 &  41.347 & 188.086 &  58.372 \\ 
   \midrule
   Alle & 466.900 & 107.614 & 741.646 & 157.957 & 946.665 & 168.888 \\ 
   \bottomrule
\end{tabular}
\endgroup
\caption{Vorr\"ate und Fehler in Kubikmetern Derbholz mit Rinde.} 
\end{table}

\subsubsection{Alle Baumarten}

\begin{center}\includegraphics{./devel.R_graphics/Kreis_Ostalbkreis_Privatwald_Vorrat_Alle_BA.eps}\end{center}
% latex table generated in R 3.3.3 by xtable 1.8-2 package
% Wed May 16 11:03:02 2018
\begin{table}[ht]
\centering
\begingroup\scriptsize
\begin{tabular}{l|rr|rr|rr}
  BWI &  \multicolumn{2}{c}{1987} & \multicolumn{2}{c}{2002} & \multicolumn{2}{c}{2012} \\ \toprule
Durchmesser & Vorrat & Fehler & Vorrat & Fehler & Vorrat & Fehler \\ 
  \midrule
 7-24.9 &  2.333.445 &   303.018 & 1.892.019 &   253.401 & 2.084.151 &   255.179 \\ 
  25-49.9 &  6.703.308 &   837.196 & 5.516.753 &   664.337 & 5.052.347 &   628.719 \\ 
     $>$=50 &  1.276.942 &   268.641 & 1.732.226 &   325.667 & 2.207.434 &   385.200 \\ 
   \midrule
   Alle & 10.313.696 & 1.170.469 & 9.140.997 & 1.040.259 & 9.343.933 & 1.072.634 \\ 
   \bottomrule
\end{tabular}
\endgroup
\caption{Vorr\"ate und Fehler in Kubikmetern Derbholz mit Rinde.} 
\end{table}
\newpage
\subsection{Naturn"a{}he}
Bei der Bestimmung der Naturn"a{}he wurde der Vergleichbarkeit
               wegen f"u{}r BWI 2 und BWI 3 die Zuordnung der
               nat"u{}rlichen Waldgesellschaft der BWI 3 zugrundegelegt.

\begin{center}\includegraphics{./devel.R_graphics/Kreis_Ostalbkreis_Privatwald_Naturnaehe.eps}\end{center}
% latex table generated in R 3.3.3 by xtable 1.8-2 package
% Wed May 16 11:03:04 2018
\begin{table}[ht]
\centering
\begingroup\scriptsize
\begin{tabular}{l|rr|rr}
  &  \multicolumn{2}{c}{2002} & \multicolumn{2}{c}{2012} \\ \toprule
Naturnaehestufe & Fl"achenanteil & Fehler & Fl"achenanteil & Fehler \\ 
  \midrule
   sehr naturnah & 18,7 & 3,2 & 17,7 & 3,0 \\ 
          naturnah & 25,8 & 3,0 & 25,0 & 3,0 \\ 
  bedingt naturnah & 30,7 & 3,3 & 31,9 & 3,3 \\ 
      kulturbetont &  4,6 & 1,5 &  4,9 & 1,5 \\ 
    kulturbestimmt & 18,7 & 3,4 & 19,8 & 3,4 \\ 
   \midrule
    keine Angabe &  1,4 & 0,7 &  0,7 & 0,5 \\ 
   \bottomrule
\end{tabular}
\endgroup
\caption{Naturn\"ahestufen in der Fl\"ache und Fehler in Prozent.} 
\end{table}
\subsection{Forstlich bedeutsame Arten}

\begin{center}\includegraphics{./devel.R_graphics/Kreis_Ostalbkreis_Privatwald_Forstlich_bedeutsame_Arten.eps}\end{center}
% latex table generated in R 3.3.3 by xtable 1.8-2 package
% Wed May 16 11:03:05 2018
\begin{table}[ht]
\centering
\begingroup\scriptsize
\begin{tabular}{l|rr|rr}
  BWI &  \multicolumn{2}{c}{2002} & \multicolumn{2}{c}{2012} \\ \toprule
Art & Flächenanteil & Fehler & Flächenanteil & Fehler \\ 
  \midrule
  Adlerfarn & 0,0 & 0,00 & 0,0 & 0,03 \\ 
   Brennessel & 4,0 & 0,64 & 3,1 & 0,59 \\ 
    Brombeere & 6,6 & 1,27 & 6,8 & 1,16 \\ 
   Heidekraut & 0,0 & 0,00 & 0,0 & 0,02 \\ 
  Heidelbeere & 3,8 & 0,94 & 3,9 & 0,98 \\ 
    Honiggras & 0,0 & 0,02 & 0,3 & 0,16 \\ 
     Reitgras & 0,9 & 0,41 & 1,9 & 0,55 \\ 
     Riedgras & 4,2 & 1,21 & 6,8 & 1,68 \\ 
   \bottomrule
\end{tabular}
\endgroup
\caption{Forstlich bedeutsame Arten: Anteile und Fehler in Prozent.} 
\end{table}
\clearpage
\subsection{Totholz}

\begin{center}\includegraphics[width=1\textwidth]{./devel.R_graphics/Kreis_Ostalbkreis_Privatwald_Totholz.eps}\end{center}
% latex table generated in R 3.3.3 by xtable 1.8-2 package
% Wed May 16 11:03:10 2018
\begin{table}[ht]
\centering
\begingroup\scriptsize
\begin{tabular}{l|rr|rr|rr}
  BWI &  \multicolumn{2}{c}{2002} & \multicolumn{2}{c}{2012*} & \multicolumn{2}{c}{2012} \\ \toprule
Totholzart & Vorrat & Fehler & Vorrat & Fehler & Vorrat & Fehler \\ 
  \midrule
Liegend & 6.4 & 1.3 & 5.4 & 1.3 & 8.3 & 1.5 \\ 
  Stehend (Baum) & 1.0 & 0.2 & 0.3 & 0.0 & 0.6 & 0.1 \\ 
  Stehend (Bruchstück) & 0.4 & 0.1 & 2.5 & 1.3 & 2.8 & 1.2 \\ 
  Wurzelstock & 6.1 & 0.8 & 5.8 & 0.7 & 8.6 & 0.9 \\ 
  Abfuhrrest & 0.0 &  & 0.0 &  & 0.0 &  \\ 
   \midrule
Gesamtvorrat & 13.9 & 2.5 & 13.9 & 2.2 & 20.3 & 2.4 \\ 
   \bottomrule
\end{tabular}
\endgroup
\caption{Totholz: Vorr\"ate und Fehler in Kubikmetern Derbholz mit Rinde je Hektar,
                   2012*:  Berechnung des Totholzes 2012 nach Aufnahmekriterien von 2002.} 
\end{table}
\newpage
\subsection{Zuwachs und ausgeschiedener Vorrat}
Ausgeschiedener Vorrat umfasst genutzte und ungenutzt im Wald
               verbliebene Mengen.

\begin{center}\includegraphics[width=1\textwidth]{./devel.R_graphics/Kreis_Ostalbkreis_Privatwald_Zuwachs_Abgang.eps}\end{center}
% latex table generated in R 3.3.3 by xtable 1.8-2 package
% Wed May 16 11:03:17 2018
\begin{table}[ht]
\centering
\begingroup\scriptsize
\begin{tabular}{ll|rr|rr|rr|rr|rr}
  &  \multicolumn{2}{c}{} \\ \toprule
Artengruppe & Periode & Zuwachs & Fehler & Abgang & Fehler \\ 
  \midrule
                                Fichte & 1987 - 2002 & 15,8 & 0,72 & 24,9 & 2,87 \\ 
                                  Fichte & 2002 - 2012 & 13,5 & 0,81 & 16,9 & 2,05 \\ 
  Weißtanne/Douglasie/Kiefer/Lärchen/sNB & 1987 - 2002 & 11,9 & 1,03 &  8,9 & 1,72 \\ 
  Weißtanne/Douglasie/Kiefer/Lärchen/sNB & 2002 - 2012 & 10,9 & 1,31 & 10,6 & 2,43 \\ 
                                   Buche & 1987 - 2002 & 11,2 & 0,82 &  8,6 & 1,55 \\ 
                                   Buche & 2002 - 2012 & 10,6 & 1,08 &  7,9 & 1,58 \\ 
                                  Eichen & 1987 - 2002 &  8,0 & 0,91 &  5,6 & 1,36 \\ 
                                  Eichen & 2002 - 2012 &  8,9 & 1,20 &  6,1 & 1,62 \\ 
            Esche/Bergahorn/HBu/sBlb/Aln & 1987 - 2002 &  7,5 & 1,09 &  2,9 & 0,79 \\ 
            Esche/Bergahorn/HBu/sBlb/Aln & 2002 - 2012 &  7,0 & 0,66 &  3,6 & 0,99 \\ 
   \midrule
                               Alle BA & 1987 - 2002 & 13,1 & 0,51 & 16,1 & 1,64 \\ 
                                 Alle BA & 2002 - 2012 & 11,4 & 0,53 & 11,8 & 1,17 \\ 
   \bottomrule
\end{tabular}
\endgroup
\caption{Zuwachs, ausgeschiedener Vorrat (Abgang) und Fehler in Kubikmeternn Derbholz mit Rinde pro Hektar und Jahr.} 
\end{table}

\begin{center}\includegraphics{./devel.R_graphics/Kreis_Ostalbkreis_Privatwald_Nutzungen.eps}\end{center}
% latex table generated in R 3.3.3 by xtable 1.8-2 package
% Wed May 16 11:03:27 2018
\begin{table}[ht]
\centering
\begingroup\scriptsize
\begin{tabular}{l|rr|rr|rr|rr}
  Durchmesser &  \multicolumn{2}{c}{>=50 cm} & \multicolumn{2}{c}{0-24.9 cm} & \multicolumn{2}{c}{25-49.9 cm} & \multicolumn{2}{c}{Gesamt} \\ \toprule
Periode & Nutzung & Fehler & Nutzung & Fehler & Nutzung & Fehler & Nutzung & Fehler \\ 
  \midrule
1987 - 2002 & 85.054 & 22.810 & 95.804 & 17.248 & 350.672 & 57.170 & 531.531 & 77.754 \\ 
   \midrule
2002 - 2012 & 63.920 & 13.397 & 46.460 &  8.917 & 152.994 & 24.932 & 263.373 & 37.350 \\ 
   \bottomrule
\end{tabular}
\endgroup
\caption{J\"a{}hrliche Ernte in Kubikmetern Erntevolumen ohne Rinde 
                  nach St\"a{}rkeklassen \"u{}ber alle Baumarten.} 
\end{table}
\subsection{Verj"ungung}
Als Verj"u{}ngung wird die Baumschicht bis zu einer H"o{}he
               von vier Metern betrachtet.
% latex table generated in R 3.3.3 by xtable 1.8-2 package
% Wed May 16 11:03:31 2018
\begin{table}[ht]
\centering
\begingroup\scriptsize
\begin{tabular}{lrrrrrr}
  \hline
BWI & Artengruppe & Unter Schirm & Abgedeckt & Gesamt & Fehler & NV Anteil \\ 
  \hline
2002 &                                 Fichte & 1.335 & 1.676 & 3.011,7 &   484,6 & 59,6 \\ 
  2002 & Weißtanne/Douglasie/Kiefer/Lärchen/sNB &   141 &   101 &   242,1 &    73,2 & 57,0 \\ 
  2002 &                                  Buche & 1.432 &   790 & 2.222,5 &   585,9 & 90,1 \\ 
  2002 &                                 Eichen &    11 &    59 &    70,0 &    32,0 & 31,4 \\ 
  2002 &           Esche/Bergahorn/HBu/sBlb/Aln & 1.511 & 1.064 & 2.575,6 &   443,7 & 82,0 \\ 
  2002 &                             Alle Arten & 4.431 & 3.691 & 8.122,0 & 1.125,2 & 74,8 \\ 
  2012 &                                 Fichte &   960 & 1.555 & 2.514,4 &   470,8 & 51,2 \\ 
  2012 & Weißtanne/Douglasie/Kiefer/Lärchen/sNB &   360 &   240 &   600,3 &   143,1 & 52,3 \\ 
  2012 &                                  Buche & 1.631 &   403 & 2.034,2 &   467,1 & 96,2 \\ 
  2012 &                                 Eichen &    62 &    32 &    94,1 &    32,6 & 83,0 \\ 
  2012 &           Esche/Bergahorn/HBu/sBlb/Aln & 1.660 &   591 & 2.251,3 &   369,8 & 87,8 \\ 
  2012 &                             Alle Arten & 4.673 & 2.822 & 7.494,3 & 1.008,0 & 74,9 \\ 
   \hline
\end{tabular}
\endgroup
\caption{Verj\"ungung in Hektar, Anteil der Naturverj\"ungung 
                  (NV Anteil) an der gesamten Naturverj\"ungung (Gesamt)
                  in Prozent.} 
\end{table}
\newpage
\section{Öffentlicher Wald}
\subsection{Baumartenfl\"a{}chen}
Alle Baumartenfl\"a{}chen enthalten anteilig Bl\"o{}\ss{}en
               und L\"u{}cken.
\subsubsection{Entwicklung von 1987 bis 2012}

\begin{center}\includegraphics{./devel.R_graphics/Kreis_Ostalbkreis_Oeffentlicher_Wald_Flaeche.eps}\end{center}
% latex table generated in R 3.3.3 by xtable 1.8-2 package
% Wed May 16 11:03:44 2018
\begin{table}[ht]
\centering
\begingroup\scriptsize
\begin{tabular}{l|rr|rr|rr}
  BWI &  \multicolumn{2}{c}{1987} & \multicolumn{2}{c}{2002} & \multicolumn{2}{c}{2012} \\ \toprule
Artengruppe & Fläche & Fehler & Fläche & Fehler & Fläche & Fehler \\ 
  \midrule
                                Fichte & 18.957 & 2.132 & 15.437 & 1.761 & 13.194 & 1.617 \\ 
  Weißtanne/Douglasie/Kiefer/Lärchen/sNB &  3.983 &   640 &  4.924 &   792 &  4.856 &   754 \\ 
                                   Buche &  5.529 & 1.023 &  6.640 & 1.070 &  7.593 & 1.125 \\ 
                                  Eichen &    743 &   177 &  1.153 &   262 &  1.376 &   293 \\ 
            Esche/Bergahorn/HBu/sBlb/Aln &  1.995 &   408 &  3.555 &   619 &  4.798 &   699 \\ 
   \midrule
                               Alle BA & 31.208 & 3.214 & 31.708 & 3.221 & 31.818 & 3.241 \\ 
   \bottomrule
\end{tabular}
\endgroup
\caption{Fl\"achen und Fehler in Hektar.} 
\end{table}
% latex table generated in R 3.3.3 by xtable 1.8-2 package
% Wed May 16 11:03:45 2018
\begin{table}[ht]
\centering
\begingroup\scriptsize
\begin{tabular}{l|rr|rr|rr}
  BWI &  \multicolumn{2}{c}{1987} & \multicolumn{2}{c}{2002} & \multicolumn{2}{c}{2012} \\ \toprule
Artengruppe & Anteil & Fehler & Anteil & Fehler & Anteil & Fehler \\ 
  \midrule
                                Fichte & 60,7 & 2,90 & 48,7 & 2,73 & 41,5 & 2,74 \\ 
  Weißtanne/Douglasie/Kiefer/Lärchen/sNB & 12,8 & 1,60 & 15,5 & 1,93 & 15,3 & 1,77 \\ 
                                   Buche & 17,7 & 2,65 & 20,9 & 2,56 & 23,9 & 2,57 \\ 
                                  Eichen &  2,4 & 0,52 &  3,6 & 0,75 &  4,3 & 0,83 \\ 
            Esche/Bergahorn/HBu/sBlb/Aln &  6,4 & 1,13 & 11,2 & 1,59 & 15,1 & 1,63 \\ 
   \bottomrule
\end{tabular}
\endgroup
\caption{Flächenanteil und Fehler in Prozent.} 
\end{table}
\newpage
\subsubsection{Entwicklung der Altersstruktur von 1987 bis 2012}

\subsubsection{Fichte}

\begin{center}\includegraphics{./devel.R_graphics/Kreis_Ostalbkreis_Oeffentlicher_Wald_Flaeche_Fichte.eps}\end{center}
% latex table generated in R 3.3.3 by xtable 1.8-2 package
% Wed May 16 11:03:47 2018
\begin{table}[ht]
\centering
\begingroup\scriptsize
\begin{tabular}{l|rr|rr|rr}
  BWI &  \multicolumn{2}{c}{1987} & \multicolumn{2}{c}{2002} & \multicolumn{2}{c}{2012} \\ \toprule
Alter & Fläche & Fehler & Fläche & Fehler & Fläche & Fehler \\ 
   1-60 &  9.491 & 1.313 &  9.788 & 1.289 &  7.344 & 1.114 \\ 
  61-120 &  8.998 & 1.316 &  5.210 &   868 &  4.970 &   837 \\ 
    $>$120 &    468 &   217 &    438 &   193 &    880 &   299 \\ 
   \midrule
  Alle & 18.957 & 2.132 & 15.437 & 1.761 & 13.194 & 1.617 \\ 
   \bottomrule
\end{tabular}
\endgroup
\caption{Fl\"achen und Fehler in Hektar.} 
\end{table}

\subsubsection{Weißtanne, Douglasie, Kiefer, Lärchen, Lärchen/Sonstiges Nadelholz}

\begin{center}\includegraphics{./devel.R_graphics/Kreis_Ostalbkreis_Oeffentlicher_Wald_Flaeche_Weisstanne_Douglasie_Kiefer_Laerchen_sNB.eps}\end{center}
% latex table generated in R 3.3.3 by xtable 1.8-2 package
% Wed May 16 11:03:51 2018
\begin{table}[ht]
\centering
\begingroup\scriptsize
\begin{tabular}{l|rr|rr|rr}
  BWI &  \multicolumn{2}{c}{1987} & \multicolumn{2}{c}{2002} & \multicolumn{2}{c}{2012} \\ \toprule
Alter & Fläche & Fehler & Fläche & Fehler & Fläche & Fehler \\ 
   1-60 & 1.933 & 429 & 2.387 & 510 & 2.233 & 474 \\ 
  61-120 & 1.686 & 346 & 2.098 & 478 & 1.890 & 439 \\ 
    $>$120 &   363 & 168 &   439 & 185 &   733 & 225 \\ 
   \midrule
  Alle & 3.983 & 640 & 4.924 & 792 & 4.856 & 754 \\ 
   \bottomrule
\end{tabular}
\endgroup
\caption{Fl\"achen und Fehler in Hektar.} 
\end{table}

\subsubsection{Buche}

\begin{center}\includegraphics{./devel.R_graphics/Kreis_Ostalbkreis_Oeffentlicher_Wald_Flaeche_Buche.eps}\end{center}
% latex table generated in R 3.3.3 by xtable 1.8-2 package
% Wed May 16 11:03:52 2018
\begin{table}[ht]
\centering
\begingroup\scriptsize
\begin{tabular}{l|rr|rr|rr}
  BWI &  \multicolumn{2}{c}{1987} & \multicolumn{2}{c}{2002} & \multicolumn{2}{c}{2012} \\ \toprule
Alter & Fläche & Fehler & Fläche & Fehler & Fläche & Fehler \\ 
   1-60 & 2.664 &   536 & 3.027 &   570 & 3.456 &   602 \\ 
  61-120 & 2.297 &   662 & 2.701 &   622 & 3.231 &   701 \\ 
    $>$120 &   567 &   279 &   911 &   325 &   905 &   267 \\ 
   \midrule
  Alle & 5.529 & 1.023 & 6.640 & 1.070 & 7.593 & 1.125 \\ 
   \bottomrule
\end{tabular}
\endgroup
\caption{Fl\"achen und Fehler in Hektar.} 
\end{table}

\subsubsection{Eichen}

\begin{center}\includegraphics{./devel.R_graphics/Kreis_Ostalbkreis_Oeffentlicher_Wald_Flaeche_Eichen.eps}\end{center}
% latex table generated in R 3.3.3 by xtable 1.8-2 package
% Wed May 16 11:03:53 2018
\begin{table}[ht]
\centering
\begingroup\scriptsize
\begin{tabular}{l|rr|rr|rr}
  BWI &  \multicolumn{2}{c}{1987} & \multicolumn{2}{c}{2002} & \multicolumn{2}{c}{2012} \\ \toprule
Alter & Fläche & Fehler & Fläche & Fehler & Fläche & Fehler \\ 
   1-60 & 464 & 136 &   597 & 194 &   734 & 227 \\ 
  61-120 & 198 &  90 &   333 & 116 &   503 & 155 \\ 
    $>$120 &  81 &  62 &   222 & 108 &   140 &  65 \\ 
   \midrule
  Alle & 743 & 177 & 1.153 & 262 & 1.376 & 293 \\ 
   \bottomrule
\end{tabular}
\endgroup
\caption{Fl\"achen und Fehler in Hektar.} 
\end{table}

\subsubsection{Esche, Bergahorn, Hainbuche, Sonstiges Buntlaubholz, Anderes Laubholz niederer Lebensdauer}

\begin{center}\includegraphics{./devel.R_graphics/Kreis_Ostalbkreis_Oeffentlicher_Wald_Flaeche_Esche_Bergahorn_HBu_sBlb_Aln.eps}\end{center}
% latex table generated in R 3.3.3 by xtable 1.8-2 package
% Wed May 16 11:03:57 2018
\begin{table}[ht]
\centering
\begingroup\scriptsize
\begin{tabular}{l|rr|rr|rr}
  BWI &  \multicolumn{2}{c}{1987} & \multicolumn{2}{c}{2002} & \multicolumn{2}{c}{2012} \\ \toprule
Alter & Fläche & Fehler & Fläche & Fehler & Fläche & Fehler \\ 
   1-60 & 1.599 & 344 & 3.066 & 548 & 3.995 & 617 \\ 
  61-120 &   376 & 201 &   479 & 199 &   737 & 244 \\ 
    $>$120 &    19 &  19 &    10 &  10 &    67 &  36 \\ 
   \midrule
  Alle & 1.995 & 408 & 3.555 & 619 & 4.798 & 699 \\ 
   \bottomrule
\end{tabular}
\endgroup
\caption{Fl\"achen und Fehler in Hektar.} 
\end{table}

\subsubsection{Alle Baumarten}

\begin{center}\includegraphics{./devel.R_graphics/Kreis_Ostalbkreis_Oeffentlicher_Wald_Flaeche_Alle_BA.eps}\end{center}
% latex table generated in R 3.3.3 by xtable 1.8-2 package
% Wed May 16 11:03:58 2018
\begin{table}[ht]
\centering
\begingroup\scriptsize
\begin{tabular}{l|rr|rr|rr}
  BWI &  \multicolumn{2}{c}{1987} & \multicolumn{2}{c}{2002} & \multicolumn{2}{c}{2012} \\ \toprule
Alter & Fläche & Fehler & Fläche & Fehler & Fläche & Fehler \\ 
   1-60 & 16.152 & 2.010 & 18.865 & 2.220 & 17.762 & 2.136 \\ 
  61-120 & 13.556 & 1.802 & 10.821 & 1.533 & 11.331 & 1.567 \\ 
    $>$120 &  1.500 &   471 &  2.021 &   476 &  2.726 &   549 \\ 
   \midrule
  Alle & 31.208 & 3.214 & 31.708 & 3.221 & 31.818 & 3.241 \\ 
   \bottomrule
\end{tabular}
\endgroup
\caption{Fl\"achen und Fehler in Hektar.} 
\end{table}
\newpage
\subsection{Vorr\"ate}
Alle Vorratsangaben umfassen Haupt- und
               Nebenbestandsvorr"a{}te.
\subsubsection{Entwicklung von 1987 bis 2012}

\begin{center}\includegraphics{./devel.R_graphics/Kreis_Ostalbkreis_Oeffentlicher_Wald_Vorrat.eps}\end{center}
% latex table generated in R 3.3.3 by xtable 1.8-2 package
% Wed May 16 11:03:58 2018
\begin{table}[ht]
\centering
\begingroup\scriptsize
\begin{tabular}{l|rr|rr|rr}
  BWI &  \multicolumn{2}{c}{1987} & \multicolumn{2}{c}{2002} & \multicolumn{2}{c}{2012} \\ \toprule
Artengruppe & Vorrat & Fehler & Vorrat & Fehler & Vorrat & Fehler \\ 
  \midrule
                                    Fichte &  7.957.207 & 1.018.073 & 5.442.067 &   678.768 & 4.610.380 &   599.071 \\ 
  Weißtanne
Douglasie
Kiefer
Lärchen
sNB &  1.308.706 &   211.502 & 1.572.521 &   247.078 & 1.650.490 &   256.689 \\ 
                                       Buche &  1.352.423 &   281.776 & 1.876.240 &   334.037 & 2.332.075 &   389.950 \\ 
                                      Eichen &    210.279 &    54.507 &   266.807 &    62.951 &   401.211 &    87.178 \\ 
            Esche
Bergahorn
HBu
sBlb
Aln &    269.871 &    88.683 &   528.573 &   153.439 &   820.035 &   187.777 \\ 
   \midrule
                                   Alle BA & 11.098.486 & 1.263.929 & 9.686.207 & 1.060.105 & 9.814.191 & 1.067.247 \\ 
   \bottomrule
\end{tabular}
\endgroup
\caption{Vorr\"ate und Fehler in Kubikmetern Derbholz mit Rinde.} 
\end{table}
% latex table generated in R 3.3.3 by xtable 1.8-2 package
% Wed May 16 11:03:59 2018
\begin{table}[ht]
\centering
\begingroup\scriptsize
\begin{tabular}{l|rr|rr|rr}
  BWI &  \multicolumn{2}{c}{1987} & \multicolumn{2}{c}{2002} & \multicolumn{2}{c}{2012} \\ \toprule
Artengruppe & Vorrat & Fehler & Vorrat & Fehler & Vorrat & Fehler \\ 
  \midrule
                                Fichte & 420 & 22 & 353 & 19 & 349 & 20 \\ 
  Weißtanne/Douglasie/Kiefer/Lärchen/sNB & 329 & 21 & 319 & 29 & 340 & 26 \\ 
                                   Buche & 245 & 18 & 283 & 18 & 307 & 21 \\ 
                                  Eichen & 283 & 32 & 231 & 39 & 292 & 31 \\ 
            Esche/Bergahorn/HBu/sBlb/Aln & 135 & 33 & 149 & 33 & 171 & 28 \\ 
   \midrule
                               Alle BA & 356 & 17 & 305 & 14 & 308 & 13 \\ 
   \bottomrule
\end{tabular}
\endgroup
\caption{Vorr\"ate und Fehler in Kubikmetern Derbholz mit Rinde je Hektar.} 
\end{table}
\newpage
\subsubsection{Entwicklung der Dimensionsstruktur von 1987 bis 2012}

\subsubsection{Fichte}

\begin{center}\includegraphics{./devel.R_graphics/Kreis_Ostalbkreis_Oeffentlicher_Wald_Vorrat_Fichte.eps}\end{center}
% latex table generated in R 3.3.3 by xtable 1.8-2 package
% Wed May 16 11:04:00 2018
\begin{table}[ht]
\centering
\begingroup\scriptsize
\begin{tabular}{l|rr|rr|rr}
  BWI &  \multicolumn{2}{c}{1987} & \multicolumn{2}{c}{2002} & \multicolumn{2}{c}{2012} \\ \toprule
Durchmesser & Vorrat & Fehler & Vorrat & Fehler & Vorrat & Fehler \\ 
  \midrule
 7-24.9 & 1.778.346 &   281.270 & 1.047.691 & 162.551 &   767.123 & 145.929 \\ 
  25-49.9 & 5.467.079 &   765.271 & 3.586.247 & 480.773 & 3.066.563 & 432.185 \\ 
     $>$=50 &   711.782 &   151.030 &   808.128 & 177.033 &   776.694 & 150.940 \\ 
   \midrule
   Alle & 7.957.207 & 1.018.073 & 5.442.067 & 678.768 & 4.610.380 & 599.071 \\ 
   \bottomrule
\end{tabular}
\endgroup
\caption{Vorr\"ate und Fehler in Kubikmetern Derbholz mit Rinde.} 
\end{table}

\subsubsection{Weißtanne, Douglasie, Kiefer, Lärchen, Lärchen/Sonstiges Nadelholz}

\begin{center}\includegraphics{./devel.R_graphics/Kreis_Ostalbkreis_Oeffentlicher_Wald_Vorrat_Weisstanne_Douglasie_Kiefer_Laerchen_sNB.eps}\end{center}
% latex table generated in R 3.3.3 by xtable 1.8-2 package
% Wed May 16 11:04:01 2018
\begin{table}[ht]
\centering
\begingroup\scriptsize
\begin{tabular}{l|rr|rr|rr}
  BWI &  \multicolumn{2}{c}{1987} & \multicolumn{2}{c}{2002} & \multicolumn{2}{c}{2012} \\ \toprule
Durchmesser & Vorrat & Fehler & Vorrat & Fehler & Vorrat & Fehler \\ 
  \midrule
 7-24.9 &   255.999 &  65.220 &   181.554 &  49.025 &   189.488 &  59.895 \\ 
  25-49.9 &   892.926 & 154.316 & 1.057.852 & 183.529 &   929.568 & 165.358 \\ 
     $>$=50 &   159.781 &  39.930 &   333.115 &  68.821 &   531.434 & 103.786 \\ 
   \midrule
   Alle & 1.308.706 & 211.502 & 1.572.521 & 247.078 & 1.650.490 & 256.689 \\ 
   \bottomrule
\end{tabular}
\endgroup
\caption{Vorr\"ate und Fehler in Kubikmetern Derbholz mit Rinde.} 
\end{table}

\subsubsection{Buche}

\begin{center}\includegraphics{./devel.R_graphics/Kreis_Ostalbkreis_Oeffentlicher_Wald_Vorrat_Buche.eps}\end{center}
% latex table generated in R 3.3.3 by xtable 1.8-2 package
% Wed May 16 11:04:05 2018
\begin{table}[ht]
\centering
\begingroup\scriptsize
\begin{tabular}{l|rr|rr|rr}
  BWI &  \multicolumn{2}{c}{1987} & \multicolumn{2}{c}{2002} & \multicolumn{2}{c}{2012} \\ \toprule
Durchmesser & Vorrat & Fehler & Vorrat & Fehler & Vorrat & Fehler \\ 
  \midrule
 7-24.9 &   390.769 &  81.195 &   434.809 &  83.743 &   406.028 &  72.922 \\ 
  25-49.9 &   689.189 & 171.253 &   902.139 & 196.866 & 1.161.927 & 220.544 \\ 
     $>$=50 &   272.465 &  97.840 &   539.292 & 154.043 &   764.120 & 203.580 \\ 
   \midrule
   Alle & 1.352.423 & 281.776 & 1.876.240 & 334.037 & 2.332.075 & 389.950 \\ 
   \bottomrule
\end{tabular}
\endgroup
\caption{Vorr\"ate und Fehler in Kubikmetern Derbholz mit Rinde.} 
\end{table}

\subsubsection{Eichen}

\begin{center}\includegraphics{./devel.R_graphics/Kreis_Ostalbkreis_Oeffentlicher_Wald_Vorrat_Eichen.eps}\end{center}
% latex table generated in R 3.3.3 by xtable 1.8-2 package
% Wed May 16 11:04:06 2018
\begin{table}[ht]
\centering
\begingroup\scriptsize
\begin{tabular}{l|rr|rr|rr}
  BWI &  \multicolumn{2}{c}{1987} & \multicolumn{2}{c}{2002} & \multicolumn{2}{c}{2012} \\ \toprule
Durchmesser & Vorrat & Fehler & Vorrat & Fehler & Vorrat & Fehler \\ 
  \midrule
 7-24.9 &  63.279 & 25.273 &  43.446 & 19.509 &  61.532 & 32.657 \\ 
  25-49.9 &  96.475 & 33.781 & 124.904 & 32.879 & 206.146 & 50.813 \\ 
     $>$=50 &  50.525 & 26.484 &  98.456 & 36.721 & 133.534 & 46.256 \\ 
   \midrule
   Alle & 210.279 & 54.507 & 266.807 & 62.951 & 401.211 & 87.178 \\ 
   \bottomrule
\end{tabular}
\endgroup
\caption{Vorr\"ate und Fehler in Kubikmetern Derbholz mit Rinde.} 
\end{table}

\subsubsection{Esche, Bergahorn, Hainbuche, Sonstiges Buntlaubholz, Anderes Laubholz niederer Lebensdauer}

\begin{center}\includegraphics{./devel.R_graphics/Kreis_Ostalbkreis_Oeffentlicher_Wald_Vorrat_Esche_Bergahorn_HBu_sBlb_Aln.eps}\end{center}
% latex table generated in R 3.3.3 by xtable 1.8-2 package
% Wed May 16 11:04:07 2018
\begin{table}[ht]
\centering
\begingroup\scriptsize
\begin{tabular}{l|rr|rr|rr}
  BWI &  \multicolumn{2}{c}{1987} & \multicolumn{2}{c}{2002} & \multicolumn{2}{c}{2012} \\ \toprule
Durchmesser & Vorrat & Fehler & Vorrat & Fehler & Vorrat & Fehler \\ 
  \midrule
 7-24.9 &  87.606 & 24.939 & 199.778 &  50.537 & 301.341 &  74.377 \\ 
  25-49.9 & 158.724 & 64.921 & 283.015 &  98.403 & 427.991 & 124.975 \\ 
     $>$=50 &  23.542 & 19.059 &  45.780 &  32.370 &  90.702 &  39.120 \\ 
   \midrule
   Alle & 269.871 & 88.683 & 528.573 & 153.439 & 820.035 & 187.777 \\ 
   \bottomrule
\end{tabular}
\endgroup
\caption{Vorr\"ate und Fehler in Kubikmetern Derbholz mit Rinde.} 
\end{table}

\subsubsection{Alle Baumarten}

\begin{center}\includegraphics{./devel.R_graphics/Kreis_Ostalbkreis_Oeffentlicher_Wald_Vorrat_Alle_BA.eps}\end{center}
% latex table generated in R 3.3.3 by xtable 1.8-2 package
% Wed May 16 11:04:11 2018
\begin{table}[ht]
\centering
\begingroup\scriptsize
\begin{tabular}{l|rr|rr|rr}
  BWI &  \multicolumn{2}{c}{1987} & \multicolumn{2}{c}{2002} & \multicolumn{2}{c}{2012} \\ \toprule
Durchmesser & Vorrat & Fehler & Vorrat & Fehler & Vorrat & Fehler \\ 
  \midrule
 7-24.9 &  2.575.999 &   352.056 & 1.907.278 &   238.296 & 1.725.512 &   236.603 \\ 
  25-49.9 &  7.304.393 &   909.613 & 5.954.157 &   709.306 & 5.792.195 &   680.925 \\ 
     $>$=50 &  1.218.094 &   211.622 & 1.824.772 &   292.092 & 2.296.484 &   345.857 \\ 
   \midrule
   Alle & 11.098.486 & 1.263.929 & 9.686.207 & 1.060.105 & 9.814.191 & 1.067.247 \\ 
   \bottomrule
\end{tabular}
\endgroup
\caption{Vorr\"ate und Fehler in Kubikmetern Derbholz mit Rinde.} 
\end{table}
\newpage
\subsection{Naturn"a{}he}
Bei der Bestimmung der Naturn"a{}he wurde der Vergleichbarkeit
               wegen f"u{}r BWI 2 und BWI 3 die Zuordnung der
               nat"u{}rlichen Waldgesellschaft der BWI 3 zugrundegelegt.

\begin{center}\includegraphics{./devel.R_graphics/Kreis_Ostalbkreis_Oeffentlicher_Wald_Naturnaehe.eps}\end{center}
% latex table generated in R 3.3.3 by xtable 1.8-2 package
% Wed May 16 11:04:13 2018
\begin{table}[ht]
\centering
\begingroup\scriptsize
\begin{tabular}{l|rr|rr}
  &  \multicolumn{2}{c}{2002} & \multicolumn{2}{c}{2012} \\ \toprule
Naturnaehestufe & Fl"achenanteil & Fehler & Fl"achenanteil & Fehler \\ 
  \midrule
   sehr naturnah & 16,1 & 2,5 & 16,7 & 2,6 \\ 
          naturnah & 35,0 & 2,8 & 35,8 & 2,8 \\ 
  bedingt naturnah & 38,8 & 3,1 & 38,4 & 3,2 \\ 
      kulturbetont &  4,1 & 1,3 &  4,1 & 1,3 \\ 
    kulturbestimmt &  5,7 & 1,5 &  5,0 & 1,5 \\ 
   \midrule
    keine Angabe &  0,3 & 0,3 &  0,0 & 0,0 \\ 
   \bottomrule
\end{tabular}
\endgroup
\caption{Naturn\"ahestufen in der Fl\"ache und Fehler in Prozent.} 
\end{table}
\subsection{Forstlich bedeutsame Arten}

\begin{center}\includegraphics{./devel.R_graphics/Kreis_Ostalbkreis_Oeffentlicher_Wald_Forstlich_bedeutsame_Arten.eps}\end{center}
% latex table generated in R 3.3.3 by xtable 1.8-2 package
% Wed May 16 11:04:15 2018
\begin{table}[ht]
\centering
\begingroup\scriptsize
\begin{tabular}{l|rr|rr}
  BWI &  \multicolumn{2}{c}{2002} & \multicolumn{2}{c}{2012} \\ \toprule
Art & Flächenanteil & Fehler & Flächenanteil & Fehler \\ 
  \midrule
  Adlerfarn & 0,0 & 0,00 & 0,0 & 0,05 \\ 
   Brennessel & 3,2 & 0,73 & 2,7 & 0,42 \\ 
    Brombeere & 7,9 & 1,14 & 9,5 & 1,22 \\ 
   Heidekraut & 0,0 & 0,00 & 0,1 & 0,04 \\ 
  Heidelbeere & 2,9 & 0,63 & 4,7 & 1,05 \\ 
    Honiggras & 0,0 & 0,03 & 0,3 & 0,14 \\ 
     Reitgras & 0,4 & 0,15 & 2,4 & 0,50 \\ 
     Riedgras & 2,6 & 0,62 & 4,5 & 1,11 \\ 
   \bottomrule
\end{tabular}
\endgroup
\caption{Forstlich bedeutsame Arten: Anteile und Fehler in Prozent.} 
\end{table}
\clearpage
\subsection{Totholz}

\begin{center}\includegraphics[width=1\textwidth]{./devel.R_graphics/Kreis_Ostalbkreis_Oeffentlicher_Wald_Totholz.eps}\end{center}
% latex table generated in R 3.3.3 by xtable 1.8-2 package
% Wed May 16 11:04:20 2018
\begin{table}[ht]
\centering
\begingroup\scriptsize
\begin{tabular}{l|rr|rr|rr}
  BWI &  \multicolumn{2}{c}{2002} & \multicolumn{2}{c}{2012*} & \multicolumn{2}{c}{2012} \\ \toprule
Totholzart & Vorrat & Fehler & Vorrat & Fehler & Vorrat & Fehler \\ 
  \midrule
Liegend & 10.4 & 2.4 & 11.0 & 2.6 & 14.8 & 2.7 \\ 
  Stehend (Baum) & 0.8 & 0.2 & 1.6 & 0.4 & 2.0 & 1.0 \\ 
  Stehend (Bruchstück) & 1.3 & 0.3 & 1.0 & 0.2 & 1.4 & 0.3 \\ 
  Wurzelstock & 5.7 & 0.5 & 6.6 & 0.6 & 10.5 & 0.8 \\ 
  Abfuhrrest & 0.1 & 0.0 & 0.0 &  & 0.0 &  \\ 
   \midrule
Gesamtvorrat & 18.3 & 2.7 & 20.1 & 3.0 & 28.8 & 3.1 \\ 
   \bottomrule
\end{tabular}
\endgroup
\caption{Totholz: Vorr\"ate und Fehler in Kubikmetern Derbholz mit Rinde je Hektar,
                   2012*:  Berechnung des Totholzes 2012 nach Aufnahmekriterien von 2002.} 
\end{table}
\newpage
\subsection{Zuwachs und ausgeschiedener Vorrat}
Ausgeschiedener Vorrat umfasst genutzte und ungenutzt im Wald
               verbliebene Mengen.

\begin{center}\includegraphics[width=1\textwidth]{./devel.R_graphics/Kreis_Ostalbkreis_Oeffentlicher_Wald_Zuwachs_Abgang.eps}\end{center}
% latex table generated in R 3.3.3 by xtable 1.8-2 package
% Wed May 16 11:04:27 2018
\begin{table}[ht]
\centering
\begingroup\scriptsize
\begin{tabular}{ll|rr|rr|rr|rr|rr}
  &  \multicolumn{2}{c}{} \\ \toprule
Artengruppe & Periode & Zuwachs & Fehler & Abgang & Fehler \\ 
  \midrule
                                Fichte & 1987 - 2002 & 15,6 & 0,50 & 25,3 & 2,19 \\ 
                                  Fichte & 2002 - 2012 & 12,9 & 0,59 & 19,1 & 1,74 \\ 
  Weißtanne/Douglasie/Kiefer/Lärchen/sNB & 1987 - 2002 & 12,6 & 1,16 &  9,1 & 1,42 \\ 
  Weißtanne/Douglasie/Kiefer/Lärchen/sNB & 2002 - 2012 & 11,9 & 1,17 & 11,9 & 1,89 \\ 
                                   Buche & 1987 - 2002 & 11,5 & 0,51 &  5,6 & 0,98 \\ 
                                   Buche & 2002 - 2012 & 11,7 & 0,60 &  7,8 & 1,32 \\ 
                                  Eichen & 1987 - 2002 &  9,6 & 1,27 &  5,7 & 1,66 \\ 
                                  Eichen & 2002 - 2012 & 10,7 & 1,42 &  3,1 & 1,16 \\ 
            Esche/Bergahorn/HBu/sBlb/Aln & 1987 - 2002 &  5,9 & 0,76 &  1,4 & 0,44 \\ 
            Esche/Bergahorn/HBu/sBlb/Aln & 2002 - 2012 &  7,4 & 0,71 &  2,3 & 0,68 \\ 
   \midrule
                               Alle BA & 1987 - 2002 & 13,4 & 0,41 & 16,5 & 1,37 \\ 
                                 Alle BA & 2002 - 2012 & 11,7 & 0,39 & 12,5 & 0,97 \\ 
   \bottomrule
\end{tabular}
\endgroup
\caption{Zuwachs, ausgeschiedener Vorrat (Abgang) und Fehler in Kubikmeternn Derbholz mit Rinde pro Hektar und Jahr.} 
\end{table}

\begin{center}\includegraphics{./devel.R_graphics/Kreis_Ostalbkreis_Oeffentlicher_Wald_Nutzungen.eps}\end{center}
% latex table generated in R 3.3.3 by xtable 1.8-2 package
% Wed May 16 11:04:38 2018
\begin{table}[ht]
\centering
\begingroup\scriptsize
\begin{tabular}{l|rr|rr|rr|rr}
  Durchmesser &  \multicolumn{2}{c}{>=50 cm} & \multicolumn{2}{c}{0-24.9 cm} & \multicolumn{2}{c}{25-49.9 cm} & \multicolumn{2}{c}{Gesamt} \\ \toprule
Periode & Nutzung & Fehler & Nutzung & Fehler & Nutzung & Fehler & Nutzung & Fehler \\ 
  \midrule
1987 - 2002 & 88.722 & 18.473 & 110.264 & 17.235 & 373.049 & 54.494 & 572.036 & 76.413 \\ 
   \midrule
2002 - 2012 & 69.255 & 15.191 &  53.689 &  8.857 & 198.001 & 26.099 & 320.945 & 40.397 \\ 
   \bottomrule
\end{tabular}
\endgroup
\caption{J\"a{}hrliche Ernte in Kubikmetern Erntevolumen ohne Rinde 
                  nach St\"a{}rkeklassen \"u{}ber alle Baumarten.} 
\end{table}
\subsection{Verj"ungung}
Als Verj"u{}ngung wird die Baumschicht bis zu einer H"o{}he
               von vier Metern betrachtet.
% latex table generated in R 3.3.3 by xtable 1.8-2 package
% Wed May 16 11:04:42 2018
\begin{table}[ht]
\centering
\begingroup\scriptsize
\begin{tabular}{lrrrrrr}
  \hline
BWI & Artengruppe & Unter Schirm & Abgedeckt & Gesamt & Fehler & NV Anteil \\ 
  \hline
2002 &                                 Fichte & 1.492 & 1.223 & 2.715,7 &   497,5 & 86,04 \\ 
  2002 & Weißtanne/Douglasie/Kiefer/Lärchen/sNB &   387 &   192 &   579,1 &   125,0 & 76,34 \\ 
  2002 &                                  Buche & 1.621 &   649 & 2.270,5 &   395,9 & 74,14 \\ 
  2002 &                                 Eichen &    10 &   192 &   202,0 &   103,4 &  4,95 \\ 
  2002 &           Esche/Bergahorn/HBu/sBlb/Aln & 1.510 &   764 & 2.274,5 &   422,1 & 72,96 \\ 
  2002 &                             Alle Arten & 5.021 & 3.021 & 8.041,9 & 1.043,8 & 76,24 \\ 
  2012 &                                 Fichte & 1.907 &   521 & 2.428,4 &   431,0 & 92,83 \\ 
  2012 & Weißtanne/Douglasie/Kiefer/Lärchen/sNB &   671 &   244 &   915,5 &   200,4 & 72,02 \\ 
  2012 &                                  Buche & 2.203 &   426 & 2.629,5 &   456,5 & 94,03 \\ 
  2012 &                                 Eichen &    29 &    70 &    99,1 &    48,8 & 13,13 \\ 
  2012 &           Esche/Bergahorn/HBu/sBlb/Aln & 1.663 &   579 & 2.242,3 &   409,7 & 86,75 \\ 
  2012 &                             Alle Arten & 6.474 & 1.841 & 8.314,8 & 1.072,4 & 88,33 \\ 
   \hline
\end{tabular}
\endgroup
\caption{Verj\"ungung in Hektar, Anteil der Naturverj\"ungung 
                  (NV Anteil) an der gesamten Naturverj\"ungung (Gesamt)
                  in Prozent.} 
\end{table}
